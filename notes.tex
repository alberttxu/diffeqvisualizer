\documentclass[11pt, oneside]{article}   	% use "amsart" instead of "article" for AMSLaTeX format
\usepackage{geometry}                		% See geometry.pdf to learn the layout options. There are lots.
\geometry{letterpaper}                   		% ... or a4paper or a5paper or ...
%\geometry{landscape}                		% Activate for rotated page geometry
%\usepackage[parfill]{parskip}    		% Activate to begin paragraphs with an empty line rather than an indent
\usepackage{graphicx}				% Use pdf, png, jpg, or eps§ with pdflatex; use eps in DVI mode
								% TeX will automatically convert eps --> pdf in pdflatex
\usepackage{amssymb}
\usepackage{hyperref}

\usepackage[scaled]{helvet}
\usepackage[T1]{fontenc}
\renewcommand\familydefault{\sfdefault}

\title{Linear differential equations}
\author{Albert Xu}

\begin{document}
\maketitle

I'm developing an application to visualize differential equations - \url{https://github.com/alberttxu/diffeqvisualizer}.
The purpose of this app is to help students understand what a differential equation means,
since many college courses don't provide much intuition.
This document explains the mathematical methods used in the demo examples in the app.

\section{Intro}

A differential equation describes a function in terms of its derivatives.
For example, the linear differential equation
\begin{equation}
\dot{x} = Ax
\end{equation}
says that a function $x(t)$ must equal its derivative $\dot{x}(t) = \frac{dx}{dt}$ when multiplied by $A$.
The word linear means $\dot{x}$ is expressed in terms of multiplications and additions of $x$.
Basically, there are no $x^2$ or $\sqrt{x}$ terms, additional constant terms, etc.
More generally a differential equation is defined as $\dot{x} = f(x)$,
but for now we only analyze the linear case.  \\

\noindent
Some important notes:
\begin{itemize}
  \item There are many possible solutions for a differential equation (DE).
  In equation (1), $x$ refers to any solution, not necessarily a specific one.
  \item For most complicated DEs there is no closed-form expression that can describe all possible solutions.
  \item For a particular (i.e. specific) solution $x(t)$, it is useful to think of $x$ not as a function, but as some quantity at time $t$ in a simulation.
    \subitem Example: $x$ represents the location of an object, and $\dot{x}$ represents its velocity.
  \item For a particular solution $x(t)$, the set of its past values is sometimes called a trajectory.
  \item The study of a particular solution with a starting value $x_\mathrm{initial}$ at $t = 0$ is called an initial value problem.
\end{itemize}


\section{1D case}

This is the simplest case: $A$, $x$, and $\dot{x}$ are numbers.
We'll explain the possible solutions using a simple example: $x$ is the location of some point on a number line and $\dot{x}$ is its velocity.

If $A = 0$ then $\dot{x} = 0$, which means velocity is zero for all $x$.
For any initial value of $x$, its trajectory never moves.

If $A > 0$ then trajectories will move away from the origin, since $\dot{x}$ is proportional to $x$ with positive feedback.
If $x > 0$ the trajectory will move to the right, and if $x < 0$ it will move to the left.

If $A < 0$ then trajectories will move towards the origin using similar logic.

All 3 cases can be described using the general-form solution
\begin{equation}
x(t) = ce^{tA} .
\end{equation}
However, for a particular $x(t)$ its solution is
\begin{equation}
x(t) = x_\mathrm{initial} e^{tA}
\end{equation}
because $x_\mathrm{initial}$ uniquely determines $c$.


\section{2D case}

Higher dimensional linear DEs are also known as linear dynamical systems.
This is much more interesting since there are numerous ways a 2D point can move relative to the origin.
An excellent course on this topic is taught at Stanford here: \url{https://see.stanford.edu/Course/EE263}.

I believe it is very hard to understand the dynamics here without a graphical application.
And so I encourage the reader to play around with the elements of $A$ to see how it affects the motion of trajectories.
In the rest of this section I won't go into troublesome formulas and derivations like in most college courses,
and instead only mention basic definitions and notation required to understand this section.

In 2 dimensions, $x$ and $\dot{x}$ are vectors with 2 components (a.k.a elements, entries), and $A$ is a 2x2 matrix.
We don't need fancy LaTeX to emphasize that $A$, $x$, and $\dot{x}$ are not scalars.
Equation (1),
$$ \dot{x} = Ax ,$$
still has the same meaning as before.
Here we can imagine that $x$ represents a particle's position in a 2D plane,
and $\dot{x}$ represents the direction (velocity) that the particle is traveling.
$A$ is a map from position to velocity: for a particle at position $x$, its velocity must equal $Ax$ regardless of time $t$.

The components of a $x$ describe its location along individual axes.
For example, $x_1$ is the particle's location along the horizontal axis,
and $x_2$ is its location along the vertical axis.
Likewise $\dot{x}_1$ and $\dot{x}_2$ represent individual velocity components.


\end{document}